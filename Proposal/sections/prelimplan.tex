\section{Preliminary Plan}

\begin{enumerate}
    \item We will continually do research on what is already published in terms of machine learning for housing appraisal and keep in mind that this is a topic many people have researched and focus on the specific aspects of our project that make it unique. This will help to further establish our problem statement and start thinking into the technical requirements for this project. This research will also include research into the different algorithms we could use and how we could create an algorithm that would best fit our data.
    
    \item We plan to start our research off with finding as much data as we can find to work with through websites like Kaggle, drivendata.org, kdnuggets, dataquest, and many more to find as much usable data as possible for our project. We are aiming to find diverse, abundant, and unbiased data to best use for our project. Tools for collecting this data may vary depending on where the data is coming from.
    
    \item Once we have found some ideal data to work with, we will run through this data and clean the data to get rid of unwanted aspects of the data, bad data, or outliers. We will also need to make sure that the data is formatted appropriately and is normalized. From this data, we can appropriately define our parameters.
    
    \item We will need to compare different algorithms to see what will work best with the data defined in our parameters. During this stage of the project, we will try to determine what algorithm will work best and compare algorithms more once we have the code done in Python programming language. 
    
    \item After the data is properly cleaned, we will need to have a model coded into the Python programming language to be manipulated by training. In this training, we will need to focus on the training accuracy, the validation accuracy, the training loss, and the validation loss. To visualize these training metrics, we will be taking advantage of Matplotlib in Python. This will allow us to identify trends of under-fitting or over-fitting and correct these issues within our parameters. During this stage we will be able to compare algorithms and adjust these algorithms as needed.  
    
    \item We will need to then conduct evaluations on the performance of the model after training. We should receive good evaluation results, yet still be able to fine tune any parameters to refine the model for improvement. We will also begin to analyze how we have derived theoretical properties of the algorithms.
    
    \item Once evaluations and refinements to the model are complete, we will display the results and research graphically so that the research is easily understood. The viewer should be able to understand how different factors can effect the price of housing within a specific ZIP Code. 
\end{enumerate}